\subsection{Allgemeine Hinweise}
\subsubsection{Querverweise}
Wird in der Arbeit auf andere Stellen (Bilder, Kapitel, Tabellen ...) verwiesen, so ist jeder Verweis immer über Querverweise zu realisieren. In \LaTeX\  sollten dazu z.B. die Befehle $\backslash$ref\{\} und $\backslash$label\{\} verwendet werden.

\subsubsection{Ausdrucke}
Beste Ergebnisse werden erzielt, wenn das Dokument immer auf demselben Drucker in derselben Auslösung ausgedruckt wird. Bei einem Wechsel der Druckertreiber ergeben sich sonst neue Seiten- und Zeilenumbrüche; auch bei einem Wechsel von einem 300dpi auf einen 600dpi Ausdruck entstehen erhebliche Unterschiede im gesamten Dokument. Durch völlig andere Zeilenumbrüche werden Trennungsfehler nicht erkannt; auch das Auffinden von zu korrigierenden Textpassagen wird durch unterschiedliche Ausdrucke erheblich erschwert. Für den endgültigen Ausdruck sind durch diese Abhängigkeit von einem speziellen Druckermodell geeignete Maßnahmen zur Gewährleistung der Verfügbarkeit  der Hardware zu ergreifen.

\subsubsection{Fußnoten}
Fußnoten werden in \LaTeX{} mit dem Befehl $\backslash$footnote\{\} erstellt. Dabei beginnen die einfügten Fußnoten immer mit einem Großbuchstaben und enden mit einem Punkt\footnote{Vergleiche diese Fußnote.}. Fußnoten sind Anmerkungen des Autors vorbehalten, die nicht zwingend zum Verständnis des Haupttextes erforderlich sind (somit im stringenten Argumentationsfluss des Haupttextes stören würden), jedoch für den Leser wertvolle zusätzliche Hinweise enthalten. Es kann sich dabei um Zusatzinformationen (z.B. alternative Formulierungen, Spezifika zitierter Literatur, prägnante Zitate, die im Haupttext stören würden), Erklärungen (z.B. weitere Formelinterpretationen, die jedoch vom Hauptgedankengang ablenken würden) oder Querverweise (Abschnittsverweise in der vorliegenden Arbeit oder spezifische, nicht zitierte Zusatzliteratur) handeln.

\subsubsection{Zitate}
Das Zitieren verwendeter Literatur erfolgt somit nicht in den Fußnoten, sondern im Haupttext unter Verwendung von $\backslash$citep\{\} und $\backslash$citet\{\} sowie BibTeX. Es werden direkte Zitate (d. h. Text wird wörtlich – in Anführungszeichen - übernommen; Quellennachweis ohne ‚vgl.‘) und indirekte Zitate (d. h. sinngemäße Wiedergabe des Textes; Quellennachweis mit ‚vgl.‘) unterschieden. Bei Zitaten mit einer Länge von zwei Seiten wird die erste Seite und "`f."' angegeben, bei mehr als zwei Seiten wird "`ff."' verwendet. 

Zitate werden in \LaTeX{} mit dem Befehl $\backslash$citep\{\} bzw. $\backslash$citet\{\} erstellt. Dabei wird $\backslash$citep\{\} verwendet, um eine Textstelle als Vergleich zu markieren. Hierzu wird vor dem abschließenden Satzzeichen der Befehl $\backslash$citep\{\} eingefügt und mit der entsprechenden Quelle parametrisiert \citep{Example2019}. $\backslash$citet\{\} wird hingegen verwendet, sobald man innerhalb des Fließtexts auf eine Quelle referenzieren will. Dazu fügt man $\backslash$citet\{\} einfach in den Text ein, \glqq so wie es bereits \citet[][]{Example2019} getan haben\grqq.

\subsubsection{Akronyme}
Akronyme werden in \LaTeX{} mit den Befehlen $\backslash$ac\{\}, $\backslash$acf\{\} und $\backslash$acs\{\} erstellt. Zusätzlich muss jedes Akronym definiert werden. Dazu wird in \lstinline{20_lists.tex} eine neue Zeile $\backslash$acro\{\}[]\{\} in der Acronym Umgebung eingefügt. Der erste Parameter beschreibt den Bezeichner für das Akronym über den man später im Text auf das Akronym verweisen kann. Der zweite Parameter stellt das eigentliche Akronym dar und der dritte Parameter ist die ausgeschriebene Langform des Akronyms. Über die drei Befehle $\backslash$ac\{\}, $\backslash$acf\{\} und $\backslash$acs\{\} lassen sich die definierten Akronyme dann im Text verwenden. $\backslash$ac\{\} erzeugt bei erster Verwendung des \ac{acro} die Langform und in Klammern das zugehörige Kürzel, bei jeder weiteren Verwendung erzeugt $\backslash$ac\{\} nur die Kurzform des \ac{acro}. $\backslash$acf\{\} erzwingt die Langform des \acf{acro} auch bei vorheriger Verwendung im Dokument. Äquivalent dazu erzwingt $\backslash$acs\{\} die Kurzform auch wenn das \acs{acro} bisher im Dokument nicht eingeführt wurde.