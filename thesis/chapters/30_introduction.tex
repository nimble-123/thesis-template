\section{Einleitung}
In diesem Abschnitt der Arbeit wird das Ziel formuliert, in einen größeren Zusammenhang eingeordnet und gegen andere Themen abgegrenzt. Die wichtigsten Begriffe des Themas müssen in der Einleitung präzise definiert werden; eine sorgfältige Formulierung ist hier besonders wichtig. Weiterhin können Hinweise zur verwendeten Untersuchungsmethodik gegeben werden. Durch die Darstellung des Gangs der Untersuchung kann auch die Zweckmäßigkeit der gewählten Gliederung hervorgehoben werden.  Nach Möglichkeit sollte dieses Kapitel nicht ‚Einleitung‘ heißen, sondern einen sinnvollen Titel mit Bezug zur Arbeit tragen.

Das einleitende Kapitel sollte also eine Hinführung zum Thema, das Ziel der Arbeit und den Aufbau der Arbeit enthalten. Diese Ausführungen basieren auf der vom jeweiligen Diplomanden anzufertigenden Disposition.

Die Erfahrung zeigt, dass ein Teil der Einleitung erst zum Schluss der Arbeit ausformuliert  werden sollte. So werden wiederholte Änderungen am Text vermieden.

Zum prinzipiellen Ablauf eines Diplomarbeitsvorhabens:

\begin{itemize}
  \item Der Diplomand setzt sich mit dem Betreuer in Verbindung.
  \item Nach maximal zwei Vorgesprächen erstellt der Diplomand eine Disposition/Proposal  und reicht diese bei seinem Betreuer ein. Die Disposition sollte ungefähr zwei Seiten Umfang haben, das Thema erläutern, das Ziel der Arbeit beschreiben und den geplanten Aufbau darlegen.
  \item Zur eigenen Hilfestellung hat der Diplomand einen Terminplan anzugeben. Dieser enthält neben angestrebten Abgabetermin entsprechende Meilensteine (z. B. Literaturrecherche beendet; Funktionsmodellierung beendet; Prototyp fertig etc.). Die jeweiligen Meilensteine unterscheiden sich naturgemäß von Arbeit zu Arbeit. Der Terminplan kann dem Diplomanden zur Kontrolle dienen, inwieweit seine Abschätzungen bezüglich der Dauer bestimmter Tätigkeiten mit dem Ist übereinstimmen und daraus u. U. Korrekturen in der weiteren Vorgehensweise vornehmen (natürlich immer in Absprache mit dem Betreuer).
  \item Wird die Disposition angenommen, kann die Diplomarbeit angemeldet werden.
  \item Die Bearbeitungsdauer für Diplomarbeiten  richtet sich nach der zugrunde zu legenden Diplomprüfungsordnung.
  \item Der maximalen Seitenumfänge des reinen Textes (ohne Verzeichnisse und Anhang) betragen:
  \begin{itemize}
    \item bei Diplomarbeiten 100 Seiten,
    \item bei Individuellen Projekten/Bachelorarbeit 80 Seiten.
    \item Von Diplomarbeiten und Individuellen Projekten/Bachelorarbeiten ist jeweils ein digitales Exemplar beim Aufgabensteller abzugeben. Für die Abgabe gedruckter Exemplare gilt die Abgabe von 3 Exemplaren an das Prüfungsamt
  \end{itemize}
  \item Sowohl Diplomarbeiten als auch Individuelle Projekte/Bachelorarbeiten sind im Rahmen eines Kolloquiums zu verteidigen.
\end{itemize}

\subsection{Motivation}
\textcolor{red}{\blindtext}

\subsection{Problemstellung}
\textcolor{red}{\blindtext}

\subsection{Lösungsansatz}
\textcolor{red}{\blindtext}

\subsection{Aufbau der Arbeit}
\textcolor{red}{\blindtext}