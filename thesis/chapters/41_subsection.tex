\subsection{Allgemeine Hinweise zur Formatierung}
Das Mischen von alter und neuer Rechtschreibung ist unzulässig. 

Für die Erstellung der eigenen Arbeit kann es sinnvoll sein, dieses Dokument zu übernehmen  und kontinuierlich die beispielhaften Bereiche gegen die eigenen neuen Passagen zu ersetzen; so bleibt der Aufbau erhalten und man verliert nicht versehentlich Formatierungen o. ä. Bei der Erstellung der Gliederung der eigenen wissenschaftlichen Arbeit sollten die beiden Kriterien Vollständigkeit und Überschneidungsfreiheit beachtet werden! Auf jeder eröffneten Gliederungsebene müssen jeweils mindestens zwei Gliederungspunkte existieren, also nicht:

\noindent--------\\
2	Ist-Zustand\\
2.1	Ist-Zustand im Unternehmen XYZ\\
3	Soll-Konzept\\
--------

Abkürzungen im Plural (Formatvorlagen) erhalten kein nachgestelltes "`s"'. Abkürzungen wie "`PCs"' oder "`CD-ROMs"' sind unzulässig.

Sollen einzelne Wörter im Text hervorgehoben werden, so ist eine kursive Hervorhebung dem Druck in fetter Schrift  vorzuziehen.
